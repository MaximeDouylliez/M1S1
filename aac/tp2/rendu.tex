\documentclass[a4paper,12pt]{report}
\usepackage[utf8]{inputenc}
\usepackage[francais]{babel}
\usepackage{fancyhdr}
\usepackage{graphicx}
\usepackage{tikz}
\usetikzlibrary{arrows}
\usetikzlibrary{calc}
\definecolor{grey}{rgb}{0.9,0.9,0.9}
\usepackage{titlesec}
\usepackage{listings}
\usepackage{textcomp}
\usepackage{hyperref}
\usepackage{ amssymb }
\usepackage{xcolor}
\usepackage{listings}
\usepackage{amsmath}
\lstset{basicstyle=\ttfamily,
  showstringspaces=false,
  commentstyle=\color{red},
  keywordstyle=\color{blue}
}

\frenchbsetup{StandardLists=true}
\newcommand{\marge}{18mm}
\usepackage[left=\marge,right=\marge,top=\marge,bottom=\marge]{geometry}
\pagestyle{fancy}
\setlength{\headheight}{14pt}
\chead{
  \textbf{Binome :} Danvin Florian$\&$ Douylliez Maxime
  \hspace{2em}
  \textbf{Groupe :} M1 Info groupe 4}
\renewcommand{\headrulewidth}{0pt}
\linespread{1.3}
\setlength{\columnseprule}{0.2pt}

\usetikzlibrary{matrix,arrows,decorations.pathmorphing}

\begin{document}
\section*{README}

Le code non symétrique se trouve dans nosym.c\\
\begin{lstlisting}[language=bash]
./test 10 7 7 3
\end{lstlisting}

\section*{Question 1}


\section*{Question 2}
(max(resultat negatif)-1)*-(1), et, (max(resultat positif)+1)*(-1)

\section*{Question 3}
\begin{itemize}
	\item 10 7 7 3: moins d'une seconde (resultat :10)
	\item 10 7 5 3: moins d'une seconde (resultat :12)
\end{itemize}

\section*{Question 4}

Lors de la création de la version dynamique, beaucoup de temps on été passé sur la représentation de la configuration en un entier. Nous n'avons pas compris qu'il s'agissait de faire de la manipulation de binaires, ce pourquoi le fichier de code contient des fonctions permetant de passer de la configuration à l'entier, et de l'entier  à la configuration en fonction du niveau d'optimisation (O1,O2,O3).Les fonctions configuration->entier sont nommé cti et les entier->configuration itc. Ces fonctions sont correcte et permette d'effectuer une correspondance entre une configuration et une valeur entiere , assurant également que le moin d'espace memoire soit utilisé pour stocker les résultats. Neamoins, au cours des executions, ces fonctions de corespondance demande trop de puissance de calcul et augmente largement la durée d'execution. Elles ne sont pas utilisé pour le reste du tp, mais o3_choco_1d les utilise.


\begin{itemize}
	\item 100 100 50 50: 197
	\item 100 100 48 52: 197
\end{itemize}

\newpage

\section*{Question 5}
Nous obtenons 128 pour ./test 127 127 0 63 et pour les autre valeur 251. Notre algo n'est pas bon.


\section*{Question 6}


\section*{Question 7}
Ces configurations ont la même valeur car la tablette de chocolat peut etre retourné, ou pivoté. 1 situation dans la réalité peut etre représenté par 8 configuration dans le programme, en fonction du point de vue.

\section*{Question 8}

la fonction utilisant les symetries s'appelle o3_choco_4d. Nous ne pouvons pas comparer les temps d'éxecutions car la fonction non optimisé, o1_choco ne rend pas un résultat similaire.


\section*{Question 9}

La fonction o3_choco_4d et o3_choco_1d prennent en compte les symétrie, en prenant une plaquette toujours plus haute que large et place le chocolat de la mort le plus en haut a gauche possible.

\section*{Question 10}
Cette question n'a pas été traité.
\end{document}